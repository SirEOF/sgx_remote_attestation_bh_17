\documentclass[11pt]{article}
\usepackage{fancyhdr}
\usepackage{url}
\usepackage{tikz}
\usepackage{hyperref}
\topmargin=-5mm
\evensidemargin=0cm
\oddsidemargin=0cm
\textwidth=16cm
\textheight=22cm
\addtolength{\headheight}{1.6pt}

\newcommand{\lastupdate}{\today}

% \lhead{\sc On the insufficiency of Intel SGX Remote Attestation}
% \rhead{\sc \lastupdate}

\title{\bf Intel SGX Remote Attestation is not sufficient for Cloud Environments}
\author{\textsc{Yogesh Prem Swami}}

\date{\lastupdate}

\begin{document}
\pagenumbering{arabic}

\maketitle

\section{Introduction}
    Intel SGX enclaves provide hardware enforced confidentiality and
    integrity guarantees for running pure code (\textit{i.e.},
    side-effect free code) in the cloud environment. By limiting the
    application's Trusted Computing Based (TCB) to the CPU and
    CPU-Cache, SGX provides security guarantees against malicious OS
    kernel and other supervisor software. This guarantee, however, is
    conditional upon the fact that the Cloud Service provider has
    indeed instantiated the enclave on a Genuine Intel SGX hardware
    and not just on a software simulator of SGX. To guard against such
    attacks, SGX provides remote attestation to validate that a
    claimed SGX enclave is indeed running on a real hardware.

    In this paper we argue that this \textit{static} remote
    attestation mechanism, while necessary, is \textit{not sufficient}
    to guarantee confidentiality and integrity of running applications
    in the cloud. In particular, we demonstrate cases where:

    \begin{itemize}
        \item A dishonest service provider instantiates both a valid
          enclave running on real hardware as well as the same enclave
          running in a software simulator in parallel, is always able
          to respond correctly to Remote Attestation queries, all the
          while running the enclave inside a software simulator with
          full access to enclave's internal state. Schemes such as
          \cite{Haven, Graphene, Scone}, which try to run unmodified
          applications inside an enclave are especially vulnerable to
          such attacks.

        \item A dishonest service provider rewinds the ``enclave's
          tape" and replays computation even though the data is
          encrypted with platform specific
          \textit{seal-keys}\cite{sgxattest}. This is a form of replay
          attack, but unlike traditional replay attacks where replay
          only affects local data, a replayed computation can affect
          remote data that are outside the realm of the service
          provider itself. For example, consider a case where an
          enclave saves encrypted credit cards on disk which cannot be
          decrypted outside of the enclave. Unless the SGX enclave
          explicitly protects itself against such attacks, which is
          non-trivial since SGX does not have either access control or
          a native replay protection mechanism, a malicious service
          provider could use the enclave as a (Turing) oracle and
          replay computation (e.g., re-purchase something using the
          credit card on a remote side) even though it cannot itself
          decrypt credit card data. We further note that since SGX can
          only handle pure\footnote{The only exception is
            \texttt{rdrand} family of instructions}, side-effect free
          computation, there's no easy mechanism to protect against
          such attacks. In fact, even if a platform has TPM, since SGX
          enclave must communicate through the operating system to
          interact with the TPM, there's no easy mechanism to
          ''build-in" replay protection within an enclave. (Intel SDK
          does provide Intel ME based replay protection---details of
          which are missing from their documentation. However, by
          trusting Intel ME, you are also implicitly trusting the
          cloud service provider, and don't need Intel SGX in that
          case.)

        \item A dishonest service provider runs multiple instances of
          the same enclave in parallel (but with potentially different
          private coins and internal state of each enclave) and
          chooses a reply that's beneficial to the attacker. For
          example, consider the case where an enclave wants to do
          Diffie-Hellman key-exchange with a remote site. Normally,
          when the remote site sends $g^x$ parameter, the local site
          has limited control in choosing $g^y$ parameters. However,
          in a cloud environment, since the cloud service provider can
          instantiate millions of copies of the enclave in parallel,
          it has greater freedom in choosing $g^y$. In particular, if
          certain properties of $x$ and $y$ can be inferred $(g^x,
          g^y)$ that provides even a small advantage to the attacker,
          that can be sufficient to break certain protocols that are
          based on Decisional Diffie-Hellman assumption. While most
          modern protocols are designed to protect against such
          adaptively chosen cipher-texts (the example above if
          completely contrived), implementation flaws in protocols as
          well as problems with random-number generators, can make
          such attacks lucrative targets.

    \end{itemize}

    In addition to bringing awareness to these attacks, this talk will
    also discuss the details about Remote Attestation mechanism. In
    particular:

    \begin{itemize}

        \item What keys are embedded inside each SGX hardware, and
          what's the protocol for providing proof of knowledge of
          these keys to remote entities? Are these protocols
          zero-knowledge, as claimed by Intel?

        \item How the EPID's \cite{epid} zero-knowledge proof of
          knowledge works, what anonymity guarantees it provides, and
          can it be replaced with other simpler schemes where platform
          anonymity is not a concern.

        \item What key-exchanges take place between Intel Attestation
          Service\footnote{
            \url{https://software.intel.com/sites/default/files/managed/7e/3b/ias-api-spec.pdf}
          }, Software Vendor's own service, Intel Provide Platform
          Enclaves (\textit{e.g.}, launch enclave
          \texttt{lible.signed.so}, etc.), and the enclave itself. To
          the best of our knowledge, no detailed analysis of this
          highly convoluted process exists in the literature.

    \end{itemize}

    Based on the attacks described above and the analysis of Remote
    Attestation, we provide necessary and sufficient conditions that
    each enclave must possess---in addition to remote attestation---to
    be secure! In spite of Intel's commendable engineering fete in
    creating SGX, we are pessimistic that large pieces of existing
    software can easily be modified to run securely in cloud
    environment without significant changes. In particular, we
    emphasize that recent schemes such as \cite{Haven, Graphene,
      Scone} which try to run unmodified user applications inside an
    enclave cannot be made unconditionally secure (to give a trivial
    example, an existing application that uses \texttt{/dev/random} to
    generate its private coins will be completely insecure if run
    without modifications inside an enclave). Furthermore, unlike Iago
    attack \cite{iago}, which assumes that enclaves are securely
    instantiated but the attacker is trying to abuse the system call
    interface post instantiation, these attacks are targeted towards
    enclave instantiation itself.

\bibliographystyle{alpha}
\bibliography{sgx_biblio}

\end{document}
